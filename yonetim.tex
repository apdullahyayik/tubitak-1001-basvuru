\begin{landscape}
\thispagestyle{mylandscape} %Call our predefined page type
\noindent{\textbf{3. PROJE YÖNETİMİ}}

\vspace*{0.1in}
\noindent\textbf{3.1 Yönetim Düzeni: İş Paketleri (İP), Görev Dağılımı ve Süreleri}

\vspace*{0.1in}
\noindent\textbf{3.1.1. İş-Zaman Çizelgesi}

\vspace*{0.1in}
Projede yer alacak başlıca iş paketlerinin hangi sürede gerçekleştirileceği ``İş-Zaman Çizelgesi'' doldurularak verilir. Literatür taraması, gelişme ve sonuç raporu hazırlama aşamaları, proje sonuçlarının paylaşımı, makale yazımı ve malzeme alımı iş paketi olarak gösterilmemelidir.

\begin{center}
\textbf{İŞ-ZAMAN ÇİZELGESİ (*)    }
\end{center}

%\renewcommand{\arraystretch}{0.8}

\begin{center}
\footnotesize
\resizebox{1.6\textwidth}{!}{%
\begin{tabular}{|l|l|l|l|l|l|l|l|l|l|l|l|l|l|l|l|l|l|l|l|l|l|l|l|l|l|l|l|l|l|l|l|l|l|l|l|l|l|l|l|}
\hline
\rowcolor[HTML]{C0C0C0} 
\multicolumn{1}{|c|}{\cellcolor[HTML]{C0C0C0}} & \multicolumn{1}{c|}{\cellcolor[HTML]{C0C0C0}} & \multicolumn{1}{c|}{\cellcolor[HTML]{C0C0C0}} & \multicolumn{1}{c|}{\cellcolor[HTML]{C0C0C0}} & \multicolumn{36}{c|}{\cellcolor[HTML]{C0C0C0}} \\
\rowcolor[HTML]{C0C0C0} 
\multicolumn{1}{|c|}{\cellcolor[HTML]{C0C0C0}} & \multicolumn{1}{c|}{\cellcolor[HTML]{C0C0C0}} & \multicolumn{1}{c|}{\cellcolor[HTML]{C0C0C0}} & \multicolumn{1}{c|}{\cellcolor[HTML]{C0C0C0}} & \multicolumn{36}{c|}{\multirow{-2}{*}{\cellcolor[HTML]{C0C0C0}\textbf{AYLAR}}} \\ \cline{5-40} 
\rowcolor[HTML]{C0C0C0} 
\multicolumn{1}{|c|}{\multirow{-3}{*}{\cellcolor[HTML]{C0C0C0}\textbf{\begin{tabular}[c]{@{}c@{}}İP \\ No\end{tabular}}}} & \multicolumn{1}{c|}{\multirow{-3}{*}{\cellcolor[HTML]{C0C0C0}\textbf{\begin{tabular}[c]{@{}c@{}}İş Paketi\\ Adı\end{tabular}}}} & \multicolumn{1}{c|}{\multirow{-3}{*}{\cellcolor[HTML]{C0C0C0}\textbf{\begin{tabular}[c]{@{}c@{}}Projenin\\ Başarısındaki\\ Önemi (\%)**\end{tabular}}}} & \multicolumn{1}{c|}{\multirow{-3}{*}{\cellcolor[HTML]{C0C0C0}\textbf{\begin{tabular}[c]{@{}c@{}}Kim(ler) \\ Tarafından \\ Gerçekleştirileceği (***)\end{tabular}}}} & \multicolumn{1}{c|}{\cellcolor[HTML]{C0C0C0}\textbf{1}} & \multicolumn{1}{c|}{\cellcolor[HTML]{C0C0C0}\textbf{2}} & \multicolumn{1}{c|}{\cellcolor[HTML]{C0C0C0}\textbf{3}} & \multicolumn{1}{c|}{\cellcolor[HTML]{C0C0C0}\textbf{4}} & \multicolumn{1}{c|}{\cellcolor[HTML]{C0C0C0}\textbf{5}} & \multicolumn{1}{c|}{\cellcolor[HTML]{C0C0C0}\textbf{6}} & \multicolumn{1}{c|}{\cellcolor[HTML]{C0C0C0}\textbf{7}} & \multicolumn{1}{c|}{\cellcolor[HTML]{C0C0C0}{\color[HTML]{000000} \textbf{8}}} & \multicolumn{1}{c|}{\cellcolor[HTML]{C0C0C0}{\color[HTML]{000000} \textbf{9}}} & \multicolumn{1}{c|}{\cellcolor[HTML]{C0C0C0}{\color[HTML]{000000} \textbf{\begin{tabular}[c]{@{}c@{}}1\\ 0\end{tabular}}}} & \multicolumn{1}{c|}{\cellcolor[HTML]{C0C0C0}{\color[HTML]{000000} \textbf{\begin{tabular}[c]{@{}c@{}}1\\ 1\end{tabular}}}} & \multicolumn{1}{c|}{\cellcolor[HTML]{C0C0C0}{\color[HTML]{000000} \textbf{\begin{tabular}[c]{@{}c@{}}1\\ 2\end{tabular}}}} & \multicolumn{1}{c|}{\cellcolor[HTML]{C0C0C0}{\color[HTML]{000000} \textbf{\begin{tabular}[c]{@{}c@{}}1\\ 3\end{tabular}}}} & \multicolumn{1}{c|}{\cellcolor[HTML]{C0C0C0}{\color[HTML]{000000} \textbf{\begin{tabular}[c]{@{}c@{}}1\\ 4\end{tabular}}}} & \multicolumn{1}{c|}{\cellcolor[HTML]{C0C0C0}{\color[HTML]{000000} \textbf{\begin{tabular}[c]{@{}c@{}}1\\ 5\end{tabular}}}} & \multicolumn{1}{c|}{\cellcolor[HTML]{C0C0C0}{\color[HTML]{000000} \textbf{\begin{tabular}[c]{@{}c@{}}1\\ 6\end{tabular}}}} & \multicolumn{1}{c|}{\cellcolor[HTML]{C0C0C0}{\color[HTML]{000000} \textbf{\begin{tabular}[c]{@{}c@{}}1\\ 7\end{tabular}}}} & \multicolumn{1}{c|}{\cellcolor[HTML]{C0C0C0}\textbf{\begin{tabular}[c]{@{}c@{}}1\\ 8\end{tabular}}} & \multicolumn{1}{c|}{\cellcolor[HTML]{C0C0C0}\textbf{\begin{tabular}[c]{@{}c@{}}1\\ 9\end{tabular}}} & \multicolumn{1}{c|}{\cellcolor[HTML]{C0C0C0}\textbf{\begin{tabular}[c]{@{}c@{}}2\\ 0\end{tabular}}} & \multicolumn{1}{c|}{\cellcolor[HTML]{C0C0C0}\textbf{\begin{tabular}[c]{@{}c@{}}2\\ 1\end{tabular}}} & \multicolumn{1}{c|}{\cellcolor[HTML]{C0C0C0}\textbf{\begin{tabular}[c]{@{}c@{}}2\\ 2\end{tabular}}} & \multicolumn{1}{c|}{\cellcolor[HTML]{C0C0C0}\textbf{\begin{tabular}[c]{@{}c@{}}2\\ 3\end{tabular}}} & \multicolumn{1}{c|}{\cellcolor[HTML]{C0C0C0}\textbf{\begin{tabular}[c]{@{}c@{}}2\\ 4\end{tabular}}} & \multicolumn{1}{c|}{\cellcolor[HTML]{C0C0C0}\textbf{\begin{tabular}[c]{@{}c@{}}2\\ 5\end{tabular}}} & \multicolumn{1}{c|}{\cellcolor[HTML]{C0C0C0}\textbf{\begin{tabular}[c]{@{}c@{}}2\\ 6\end{tabular}}} & \multicolumn{1}{c|}{\cellcolor[HTML]{C0C0C0}\textbf{\begin{tabular}[c]{@{}c@{}}2\\ 7\end{tabular}}} & \multicolumn{1}{c|}{\cellcolor[HTML]{C0C0C0}\textbf{\begin{tabular}[c]{@{}c@{}}2\\ 8\end{tabular}}} & \multicolumn{1}{c|}{\cellcolor[HTML]{C0C0C0}\textbf{\begin{tabular}[c]{@{}c@{}}2\\ 9\end{tabular}}} & \multicolumn{1}{c|}{\cellcolor[HTML]{C0C0C0}\textbf{\begin{tabular}[c]{@{}c@{}}3\\ 0\end{tabular}}} & \multicolumn{1}{c|}{\cellcolor[HTML]{C0C0C0}\textbf{\begin{tabular}[c]{@{}c@{}}3\\ 1\end{tabular}}} & \multicolumn{1}{c|}{\cellcolor[HTML]{C0C0C0}\textbf{\begin{tabular}[c]{@{}c@{}}3\\ 2\end{tabular}}} & \multicolumn{1}{c|}{\cellcolor[HTML]{C0C0C0}\textbf{\begin{tabular}[c]{@{}c@{}}3\\ 3\end{tabular}}} & \multicolumn{1}{c|}{\cellcolor[HTML]{C0C0C0}\textbf{\begin{tabular}[c]{@{}c@{}}3\\ 4\end{tabular}}} & \multicolumn{1}{c|}{\cellcolor[HTML]{C0C0C0}\textbf{\begin{tabular}[c]{@{}c@{}}3\\ 5\end{tabular}}} & \multicolumn{1}{c|}{\cellcolor[HTML]{C0C0C0}\textbf{\begin{tabular}[c]{@{}c@{}}3\\ 6\end{tabular}}} \\ \hline
 &  &  &  &  &  &  &  &  &  &  &  &  &  &  &  &  &  &  &  &  &  &  &  &  &  &  &  &  &  &  &  &  &  &  &  &  &  &  &  \\ \hline
 &  &  &  &  &  &  &  &  &  &  &  &  &  &  &  &  &  &  &  &  &  &  &  &  &  &  &  &  &  &  &  &  &  &  &  &  &  &  &  \\ \hline
 &  &  &  &  &  &  &  &  &  &  &  &  &  &  &  &  &  &  &  &  &  &  &  &  &  &  &  &  &  &  &  &  &  &  &  &  &  &  &  \\ \hline
 &  &  &  &  &  &  &  &  &  &  &  &  &  &  &  &  &  &  &  &  &  &  &  &  &  &  &  &  &  &  &  &  &  &  &  &  &  &  &  \\ \hline
 &  &  &  &  &  &  &  &  &  &  &  &  &  &  &  &  &  &  &  &  &  &  &  &  &  &  &  &  &  &  &  &  &  &  &  &  &  &  &  \\ \hline
 &  &  &  &  &  &  &  &  &  &  &  &  &  &  &  &  &  &  &  &  &  &  &  &  &  &  &  &  &  &  &  &  &  &  &  &  &  &  &  \\ \hline
\end{tabular}%
}

\end{center}





{\footnotesize
\noindent(*) Çizelgedeki satırlar gerektiği kadar genişletilebilir ve çoğaltılabilir.\\
\noindent(**) Sütun toplamı 100 olmalıdır.\\
\noindent(***) İP'de görev alacak kişilerin isimleri ve görevleri (araştırmacı, danışman, bursiyer ve yardımcı personel) yazılır. Bu aşamada bursiyer(ler)in isimlerinin belirtilmesi zorunlu değildir.
}

\end{landscape}

\noindent\textbf{3.1.2. İş Paketleri}

%Proje, izlenebilir ve ölçülebilir hedefleri olan İP'lerden oluşur. İP oluşturulurken birbirileri ile ilişkili görevlerin bir araya getirilmesi beklenir. İP'nin başarılı bir şekilde tamamlanma durumunun izlenebilmesi için her bir İP'nin hedefi, başarı ölçütü ve ara çıktısı/çıktıları somut bir şekilde belirtilir.

%Aşağıdaki İP Tablosu, her bir İP için ayrı ayrı hazırlanır. İP sayısına göre tablo çoğaltılabilir. 

\begin{tabular}{|l|l|}
\hline
\multicolumn{2}{|c|}{\cellcolor[HTML]{C0C0C0}\textbf{İŞ PAKETİ TABLOSU}} \\ \hline
\textbf{İP No: 1} & \textbf{İP Adı:} \\ \hline
\multicolumn{2}{|l|}{\textbf{İP Hedefi:}} \\ \hline
\begin{tabular}[c]{@{}l@{}}\textbf{İP Kapsamında Yapılacak İşler/Görevler:}\\ \\ \\ 
1.1\\ 
1.2.\\ 
1.3.
\end{tabular} & \begin{tabular}[c]{@{}l@{}}\textbf{Kim(ler) Tarafından Gerçekleştirileceği(*)}\\ \\ \\ 
1.1.\\ 
1.2.\\ 
1.3.
\end{tabular} \\ \hline
\multicolumn{2}{|l|}{\begin{tabular}[c]{@{}l@{}}\textbf{İP'nin Başarı Ölçütü:} \\ 
%Başarı ölçütü olarak her bir iş paketinin hangi kriterleri sağladığında başarılı sayılacağı ölçülebilir ve izlenebilir şekilde nitel ve/veya nicel olarak belirtilir.
\end{tabular}} \\ \hline
\multicolumn{2}{|l|}{\begin{tabular}[c]{@{}l@{}}\textbf{Ara Çıktılar:} \\ 
%İP için öngörülen ve başarı ölçütünün gerçekleşeceğini somut olarak gösteren (teknik rapor, liste, diyagram, analiz/ölçüm sonucu, algoritma, yazılım, anket formu, verim, ham veri vb.) ara çıktılara ilişkin bilgi verilir.\\ 
\\ 
1.1.\\ 
1.2.\\ 
1.3.
\end{tabular}} \\ \hline
\end{tabular} \\ %
{\footnotesize (*) İşler/Görevler'de görev alacak kişilerin isimleri ve görevleri (araştırmacı, danışman, bursiyer ve yardımcı personel) yazılır. Bu aşamada bursiyer(ler)in isimlerinin belirtilmesi zorunlu değildir.}

\vspace*{0.1in}
\noindent\textbf{3.2. Risk Yönetimi}

%Projenin başarısını olumsuz yönde etkileyebilecek riskler ve bu risklerle karşılaşıldığında projenin başarıyla yürütülmesini sağlamak için alınacak tedbirler (B Planı) ilgili iş paketleri belirtilerek ana hatlarıyla aşağıdaki Risk Yönetimi Tablosu'nda ifade edilir. Projenin araştırma sorusu ve/veya hipoteziyle ilgili yaşanabilecek riskler dikkate alınır. B planının uygulanması projenin temel hedeflerinden ve özgün değerinden sapmaya yol açmamalıdır. B planına geçilmesi durumunda yöntem değişikliğine gidiliyor ise bu durum ayrıntılandırılmalıdır. Risk öngörülmeyen iş paketleri bu bölümde yer almaz.

\begin{center}
    \textbf{RİSK YÖNETİMİ TABLOSU (*)}
\end{center}

\begin{tabular}{|c|l|l|}
\hline
\rowcolor[HTML]{C0C0C0} 
\textbf{İP No} & \multicolumn{1}{c|}{\cellcolor[HTML]{C0C0C0}\textbf{Risk(ler)in Tanımı}} & \multicolumn{1}{c|}{\cellcolor[HTML]{C0C0C0}\textbf{Alınacak Tedbir(ler) (B Planı)}} \\ \hline
1 &  &  \\ \hline
2 &  &  \\ \hline
\end{tabular} \\ %
{\footnotesize (*) Tablodaki satırlar gerektiği kadar genişletilebilir ve çoğaltılabilir.}

\vspace*{0.1in}
\noindent\textbf{3.3. Araştırma Olanakları}

%Projenin yürütüleceği kurum ve kuruluşlarda var olan ve projede kullanılacak olan altyapı/ekipman (laboratuvar, araç, makine-teçhizat, vb.) olanakları belirtilir.

\begin{center}
    \textbf{ARAŞTIRMA OLANAKLARI TABLOSU (*)}
\end{center}

\begin{tabular}{|l|l|}
\hline
\rowcolor[HTML]{C0C0C0} 
\begin{tabular}[c]{@{}l@{}}\textbf{Kuruluşta Bulunan Altyapı/Ekipman Türü, Modeli}\\ (Laboratuvar, Araç, Makine-Teçhizat, vb.)\end{tabular} & \textbf{Projede Kullanım Amacı} \\ \hline
 &  \\ \hline
 &  \\ \hline
\end{tabular} \\ %
{\footnotesize (*) Tablodaki satırlar gerektiği kadar genişletilebilir ve çoğaltılabilir.}