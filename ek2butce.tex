\setcounter{page}{1}
\textbf{EK-2: BÜTÇE ve GEREKÇESİ}

\vspace*{0.1in}

Aşağıdaki \textbf{Genel Bütçe Tablosu} ve \textbf{TÜBİTAK'tan Talep Edilen Bütçe Tablosu} eksiksiz olarak doldurulur. Genel Bütçe Tablosu'nun TÜBİTAK'tan Talep Edilen Katkı kısmındaki toplamlarla TÜBİTAK'tan Talep Edilen Bütçe Tablosundaki ana toplamların aynı olması gerekir. TÜBİTAK'tan talep edilen desteğin her bir kalemi için ayrıntılı gerekçe ve teknik bilgiler verilir. \textbf{Başvuru aşamasında proforma veya teknik şartname sunulmasına gerek bulunmamaktadır. Proje önerisinin desteklenmesi durumunda ilgili proforma veya teknik şartname talep edilecektir.} Sarf giderleri için, projede gerekliliğinin değerlendirilmesine imkân veren ayrıntıda liste verilmesi yeterlidir. Eğer varsa, öneren kuruluş veya destekleyen diğer kuruluş katkıları, bu katkıların niteliği ve miktarının açıkça belirtildiği, ilgili kuruluş yetkilisi ya da yetkilileri tarafından imzalanmış destek mektupları da projenin desteklenmesi durumunda talep edilecektir.


\begin{center}
    \textbf{GENEL BÜTÇE TABLOSU (TL) (*)}
\end{center}

\begin{center}
\footnotesize
\begin{tabular}{|l|l|l|l|l|l|l|l|l|}
\hline
\rowcolor[HTML]{C0C0C0} 
\multicolumn{1}{|c|}{\cellcolor[HTML]{C0C0C0}\textbf{Katkı Kaynağı}} & \multicolumn{1}{c|}{\cellcolor[HTML]{C0C0C0}\textbf{\begin{tabular}[c]{@{}c@{}}Makine ve\\ Teçhizat \\ Giderleri\\ (06.1 + \\ 06.3)\end{tabular}}} & \multicolumn{1}{c|}{\cellcolor[HTML]{C0C0C0}\textbf{\begin{tabular}[c]{@{}c@{}}Sarf\\ Giderleri\\ (03.2)\end{tabular}}} & \multicolumn{1}{c|}{\cellcolor[HTML]{C0C0C0}\textbf{\begin{tabular}[c]{@{}c@{}}Hizmet\\ Alımları\\ (03.5)\end{tabular}}} & \multicolumn{1}{c|}{\cellcolor[HTML]{C0C0C0}\textbf{\begin{tabular}[c]{@{}c@{}}Temsil ve \\ Tanıtma \\ Giderleri\\ (03.6)\end{tabular}}} & \multicolumn{1}{c|}{\cellcolor[HTML]{C0C0C0}\textbf{\begin{tabular}[c]{@{}c@{}}Seyahat \\ Giderleri\\ (03.3 + \\ 03.4)\end{tabular}}} & \multicolumn{1}{c|}{\cellcolor[HTML]{C0C0C0}\textbf{\begin{tabular}[c]{@{}c@{}}Bursiyer \\ Ücretleri\\ (05.4)\end{tabular}}} & \multicolumn{1}{c|}{\cellcolor[HTML]{C0C0C0}\textbf{\begin{tabular}[c]{@{}c@{}}Geçici İşçi \\ Ücretleri \\ (Yardımcı \\ Personel)\\ (01.3)\end{tabular}}} & \multicolumn{1}{c|}{\cellcolor[HTML]{C0C0C0}\textbf{TOPLAM}} \\ \hline
\cellcolor[HTML]{C0C0C0}\textbf{\begin{tabular}[c]{@{}l@{}}TÜBİTAK'tan \\ Talep Edilen \\ Katkı\end{tabular}} &  &  &  &  &  &  &  &  \\ \hline
\cellcolor[HTML]{C0C0C0}\textbf{\begin{tabular}[c]{@{}l@{}}Öneren \\ Kuruluş Katkısı\end{tabular}} &  &  &  &  &  &  &  &  \\ \hline
\cellcolor[HTML]{C0C0C0}\textbf{\begin{tabular}[c]{@{}l@{}}Destekleyen \\ Diğer Kuruluş \\ Katkısı (**)\end{tabular}} &  &  &  &  &  &  &  &  \\ \hline
\cellcolor[HTML]{C0C0C0}\textbf{TOPLAM} &  &  &  &  &  &  &  &  \\ \hline
\end{tabular} \\%
\end{center}
{\footnotesize 
(*) PTİ ve Kurum Hissesi bütçeye dahil olmayıp ayrıca TÜBİTAK tarafından hesaplanarak proje bütçesine ilave edilir.\\ 
 (**) Destekleyen Diğer Kuruluş sayısı birden fazla ise tabloya yeni satırlar eklenerek bu destekler belirtilir.
 }
\begin{center}
    \textbf{TÜBİTAK'TAN TALEP EDİLEN BÜTÇE TABLOSU}\\
    (Bu tabloda sadece TÜBİTAK'tan talep edilen desteklerin nitelikleri ve miktarları ayrıntılı ve gerekçeli olarak belirtilir.
Lütfen ilgili açıklamalara dikkat ediniz. Tablodaki satırlar ihtiyaç duyuldukça çoğaltılabilir ve yazım alanları genişletilebilir.)
\end{center}

\begin{center}
    
\begin{tabular}{|c|c|c|c|}
\hline
\multicolumn{4}{|c|}{\textbf{Makine ve Teçhizat Giderleri (06.1 + 06.3)}} \\ \hline
\footnotesize
\textbf{Adı / Markası / Modeli / Adedi} & 
\footnotesize
\textbf{Kullanım Gerekçesi} & 
\footnotesize
\textbf{Teknik Özellikler*} & 
\footnotesize
\textbf{Bedeli** (TL)} \\ \hline
 &  &  &  \\ \hline
 &  &  &  \\ \hline
 &  &  &  \\ \hline
 
\end{tabular} \\%
\end{center}
{\footnotesize 
\textbf{(*) Proforma veya teknik şartnamede ilgili makine/teçhizata ilişkin yer alan her türlü bilgi bu kısımda sunulur.} \\
(**) Türkiye temsilcisi aracılığıyla yapılmayan alımlar için alımların yurt dışı olduğu belirtilerek tüm masraflar dahil (gümrük bedeli, vergiler, nakliye) bedel yazılır. Yurt İçi alımlarda KDV dahil bedeli yazılır. Döviz cinsinden alınan proforma faturaların TL cinsinden karşılığı hesaplanırken fatura tarihindeki T.C. Merkez Bankası efektif satış kuru esas alınır ve öneride mutlaka belirtilir.
}


\begin{center}
\begin{tabular}{|c|c|c|}
\hline
\multicolumn{3}{|c|}{\textbf{Sarf Giderleri* (03.2)}} \\ \hline
\footnotesize
\textbf{Adı} & 
\footnotesize
\textbf{Kullanım Gerekçesi} & 
\footnotesize
\textbf{Bedeli** (TL)} \\ \hline
 &  &  \\ \hline
 &  &  \\ \hline
 &  &  \\ \hline
\end{tabular} \\ %
\end{center}
{\footnotesize
 (*) Sarf giderleri için, projede gerekliliğinin değerlendirilmesine imkân veren ayrıntıda liste verilmesi gerekir.\\
 (**)Türkiye temsilcisi aracılığıyla yapılmayan alımlar için alımların yurt dışı olduğu belirtilerek tüm masraflar dahil (gümrük bedeli, vergiler, nakliye) bedel yazılır. Yurt İçi alımlarda KDV dahil bedeli yazılır.
 }

\begin{center}
    \begin{tabular}{|c|c|l|c|}
\hline
\multicolumn{4}{|c|}{\textbf{Hizmet Alımları (03.5)}} \\ \hline
\footnotesize
\textbf{Hizmet Alımı Türü} & 
\footnotesize
\textbf{\begin{tabular}[c]{@{}c@{}}Nereden/Kimden\\ Alınacağı\end{tabular}} & 
\footnotesize
\textbf{Gerekçesi ve Kapsamı*} & 
\footnotesize
\textbf{Bedeli (TL)} \\ \hline
 &  &  &  \\ \hline
 &  &  &  \\ \hline
 &  &  &  \\ \hline
\end{tabular} \\ %
\end{center}
{\footnotesize
(*) Yapılacak hizmet alımının değerlendirilmesini mümkün kılacak bütün detaylar ile proforma veya teknik şartnamede yer alan her türlü bilgi bu kısımda sunulur. 
}

\begin{center}
\begin{tabular}{|l|c|l|c|}
\hline
\multicolumn{4}{|c|}{\textbf{Temsil ve Tanıtma Giderleri (Proje Çıktı ve Sonuçlarının Paylaşımı ve Yayılımı (*) Giderleri) (03.6)}} \\ \hline
\multicolumn{1}{|c|}{\footnotesize\textbf{Mahiyeti (**)}} &
\footnotesize
\textbf{\begin{tabular}[c]{@{}c@{}}Nereden/Kimden\\ Alınacağı\end{tabular}} & \footnotesize \textbf{Gerekçesi} & \footnotesize\textbf{Bedeli (TL)} \\ \hline
\footnotesize \textbf{\begin{tabular}[c]{@{}l@{}}Katılımcıların Yurt İçi \\ Seyahat Masrafları\end{tabular}} &  &  &  \\ \hline
\footnotesize \textbf{Salon Kirası} &  &  &  \\ \hline
\footnotesize \textbf{\begin{tabular}[c]{@{}l@{}}Çalıştay/Toplantılarda \\ İkram Gideri\end{tabular}} &  &  &  \\ \hline
\footnotesize \textbf{\begin{tabular}[c]{@{}l@{}}Web Sitesi \\ Giderleri\end{tabular}} & \multicolumn{1}{l|}{} &  & \multicolumn{1}{l|}{} \\ \hline
\footnotesize \textbf{\begin{tabular}[c]{@{}l@{}}Kırtasiye/Sarf Malzemesi/ \\ Baskı ve Cilt / Posta \\ Giderleri\end{tabular}} & \multicolumn{1}{l|}{} &  & \multicolumn{1}{l|}{} \\ \hline
\multicolumn{3}{|r|}{\footnotesize\textbf{Toplam}} & \multicolumn{1}{l|}{} \\ \hline
\end{tabular}\\ %
\end{center}
{\footnotesize
\noindent (*) Bu fasıl kapsamında beklenen proje çıktılarının ilgili paydaşlar ve potansiyel kullanıcılar ile paylaşılmasına yönelik yapılacak toplantı, çalıştay vb. çalışmalar için proje bütçesinde en fazla 15.000 TL'ye kadar ödenek talebinde bulunulabilir. Bu fasıldan başka fasıllara, başka fasıllardan bu fasıla aktarım yapılamaz. İlgili fasıldan harcama yapılabilmesi için Grup onayı alınır.\\
\noindent (**) \underline{Çalıştay, toplantı düzenlenmesi/kurum kuruluş ziyaretleri kapsamında yapılacak giderler:}
\begin{itemize}
    \item Yurt Dışından katılımcı davet edilemez, yurt dışına gidilemez.
    \item  Toplantının düzenlendiği ilden katılanlara gündelik ve konaklama ödemesi yapılmaz.
    \item  Katılımcılara yurt içi seyahat (uçak, tren, otobüs, feribot vb.) ekonomi sınıfı bilet ödenir, şehir dışından gelen katılımcılara gündelik ve konaklama için projelerdeki araştırmacı limitleri uygulanır. Konaklama ödemesi 1, gündelik ödemesi 2 günden fazla olamaz. 
    \item  Proje ekibinin şehir dışında tanıtım ziyaretlerine veya toplantılara gitmesi halinde projeler kapsamındaki yurt içi seyahat limitleri uygulanır. 
    \item  Projenin yürütüldüğü kuruluşa salon kirası vb. ödenmez. Salon kirası ödeneği ancak, proje yürütücüsü kuruluşta yer olmadığına ilişkin kuruluş yetkilisinin yazılı beyanı olması halinde talep edilebilir. Salon kirası için 1.000 TL'den fazla talep edilemez.
    \item  Çalıştay/toplantılarda ikram gideri 1.000' TL'den fazla olamaz.\\
\end{itemize}

\noindent \underline{Web sitesi giderleri:} 1.000 TL'den fazla olamaz.

\noindent \underline{Kırtasiye/Sarf malzemesi/Baskı ve cilt/Posta giderleri:} 500 TL'den fazla olamaz. 
}

\begin{center}
    \textbf{Yurt İçi Saha Çalışması Planı}\\
 (Satır sayısı gerektiği kadar arttırılabilir)
\end{center}

\begin{center}
\footnotesize
    \begin{tabular}{|c|c|c|c|c|c|c|c|c|c|}
\hline 
\multirow{2}{*}{\rotatebox{90}{\textbf{Seyahat}}} & \multirow{2}{*}{\textbf{\begin{tabular}[c]{@{}c@{}}Nereden\\ Nereye\\ Gidileceği\end{tabular}}} & \multirow{2}{*}{\textbf{\begin{tabular}[c]{@{}c@{}}Saha\\ Çalışmasının\\ Mahiyeti (**)\end{tabular}}} & \multirow{2}{*}{\textbf{\begin{tabular}[c]{@{}c@{}}Kişi x Gün\\ (***)\end{tabular}}} & \multicolumn{3}{c|}{\textbf{Şehirler Arası Ulaşım (****)}} & \multicolumn{3}{c|}{\textbf{Şehir İçi Ulaşım (****)}} \\ \cline{5-10} 
 &  &  &  & \textbf{\begin{tabular}[l]{@{}l@{}}Uçak/\\ Otobüs/\\ Tren/Gemi\end{tabular}} & \textbf{\begin{tabular}[c]{@{}c@{}}Taşıt\\ Kiralama\\ (gün)\end{tabular}} & \textbf{\begin{tabular}[c]{@{}c@{}}Özel/\\ Resmi/\\ Kiralık\\ Taşıt (km)\end{tabular}} & \textbf{\begin{tabular}[c]{@{}c@{}}Toplu\\ Taşıma\\ (biniş\\ sayısı)\end{tabular}} & \textbf{\begin{tabular}[c]{@{}c@{}}Taşıt\\ Kiralama\\ (gün)\end{tabular}} & \textbf{\begin{tabular}[c]{@{}c@{}}Özel/Resmi\\ /Kiralık\\ Taşıt (km)\end{tabular}} \\ \hline
1 &  &  &  &  &  &  &  &  &  \\ \hline
2 &  &  &  &  &  &  &  &  &  \\ \hline
\multicolumn{3}{|l|}{\textbf{TOPLAM}} &  &  &  &  &  &  &  \\ \hline
\end{tabular}%
\end{center}
{\footnotesize
\noindent (*) Saha çalışması için farklı bölgelere yapılacak seyahatler söz konusu ise her bir seyahat için ayrı bir satır doldurulmalı ve ayrı numara verilmelidir. Aynı bölgeye farklı zamanlarda gidilecek olması durumunda da her seyahate ait bilgiler birbirini takip eden satırlara ayrı ayrı girilmelidir.\\
\noindent (**) Bu bölümde nerede ( mahalle, okul, resmi/özel işyeri, hastane, milli/tabiat parkı, sulak alan, ormanlık alan, koruma bölgesi, doğal/tarihi sit alanı, arkeolojik kazı alanı, mağara, askeri bölge, özel bölge, tarım alanı, çiftlik, mezbaha vb.) ne yapılacağı ( anket, mülakat, örnek toplama, bilgi/belge temini, analiz vb.) belirtilir. Saha çalışmasının yeri ve yapılacak işin niteliğinin yasal/özel izin gerektirebileceği hatırlanarak TÜBİTAK ana sayfasında yer alan ``YASAL/ÖZEL İZİN BELGESİ BİLGİ NOTU ve ETİK KURUL ONAY BELGESİ BİLGİ NOTU'' nun tekrar incelenmesi önerilir.\\
\noindent (***) Bu bölümde ilgili saha çalışmasına proje ekibinden kaç kişinin kaç gün süre ile katılacağı belirtilir. \\
\noindent (****) Bu bölümde şehirler arası ve şehir içi ulaşımın hangi yolla gerçekleşeceği ilgili kısımda gidiş-dönüş olarak belirtilmelidir. Uçak/Otobüs/Tren kısmına ilgili ulaşım aracının niteliği, şehir içiyse toplam biniş sayısı; taşıt kiralama kısmına eğer seyahat taşıt kiralama yoluyla gerçekleştirilecekse kiralanacak taşıtın niteliği ve kaç gün kiralanacağı; şayet seyahat proje ekibine ait bir özel araç ya da kuruma ait resmi araç veya kiralanacak taşıtla gerçekleşecekse yakıt giderinin hesaplanabilmesi için Özel/Resmi/KiralıkTaşıt kısmına toplam kaç kilometre mesafe kat edileceği belirtilir. 
}


\begin{center}
\textbf{Yurt İçi Saha Çalışması Seyahat Giderleri (03.3)}\\
(Yurt İçi Saha Çalışması Planındaki verilerle uyumlu olacak şekilde doldurulur ve bütçelendirilir)    
\end{center}

\begin{center}
\footnotesize
    \begin{tabular}{|l|c|c|c|c|c|c|}
\hline
 & \textbf{\begin{tabular}[c]{@{}c@{}}Kişi\\ Sayısı\end{tabular}} & \textbf{\begin{tabular}[c]{@{}c@{}}Seyahat\\ Adedi (kez)\end{tabular}} & \textbf{\begin{tabular}[c]{@{}c@{}}Toplam\\ Gün\end{tabular}} & \textbf{\begin{tabular}[c]{@{}c@{}}Taşıt Kirası\\ (ücret x gün)\end{tabular}} & \textbf{\begin{tabular}[c]{@{}c@{}}Toplam\\ Katedilecek Yol\\ (km)\end{tabular}} & \textbf{TOPLAM} \\ \hline
\textbf{\begin{tabular}[c]{@{}l@{}}Şehirler Arası Seyahat \\ (uçak/otobüs/tren)\end{tabular}} &  &  & \cellcolor[HTML]{656565}{\color[HTML]{656565} } & \cellcolor[HTML]{656565}{\color[HTML]{656565} } & \cellcolor[HTML]{656565}{\color[HTML]{656565} } &  \\ \hline
\textbf{\begin{tabular}[c]{@{}l@{}}Şehir İçi Toplu Taşıma\\ (otobüs/tren/metro vb.)\end{tabular}} &  &  & \cellcolor[HTML]{656565} & \cellcolor[HTML]{656565} & \cellcolor[HTML]{656565} &  \\ \hline
\textbf{\begin{tabular}[c]{@{}l@{}}Özel/Resmi/Kiralık Taşıt ile\\ Seyahat (*)\end{tabular}} & \cellcolor[HTML]{656565} & \cellcolor[HTML]{656565} & \cellcolor[HTML]{656565} & \cellcolor[HTML]{656565} &  &  \\ \hline
\textbf{Taşıt Kirası Gideri} & \cellcolor[HTML]{656565} & \cellcolor[HTML]{656565} &  &  & \cellcolor[HTML]{656565} &  \\ \hline
\textbf{\begin{tabular}[c]{@{}l@{}}Gündelik (**)\\ (proje ekibi)\end{tabular}} &  &  &  & \cellcolor[HTML]{656565} & \cellcolor[HTML]{656565} &  \\ \hline
\textbf{\begin{tabular}[c]{@{}l@{}}Konaklama (**)\\ (proje ekibi)\end{tabular}} &  &  &  & \cellcolor[HTML]{656565} & \cellcolor[HTML]{656565} &  \\ \hline
\multicolumn{6}{|l|}{\textbf{TOPLAM (TL)}} &  \\ \hline
\end{tabular}\\%
\end{center}
{\footnotesize
\noindent (*) Özel/Resmi/Kiralık Taşıt ile yapılan seyahatlerde her 100 km. için 6 litre kurşunsuz benzin ücreti ödeneceği dikkate alınarak hesaplanır.\\
\noindent (**) 2020 yılı için gündelik bedeli 66,85 TL/gün; olarak belirlenmiştir. Konaklama bedeli ise (belgelenmesi kaydıyla) gündeliğin iki katı olarak belirlenmiştir. İaşe (yiyecek, içecek) giderleri gündelik kapsamında olduğu için ayrıca konaklama gideri olarak karşılanmaz.
}

\begin{center}
    \textbf{Yurt Dışı Saha Çalışması}
\end{center}

\noindent TÜBİTAK Yönetim Kurulunun 14/02/2019 tarihli toplantısında alınan karar ve 02/03/2019 tarihli Başkanlık duyurusu gereğince, yalnızca Sosyal ve Beşeri Bilimler alanındaki projeler kapsamında yurt dışı saha çalışmalarına aşağıda belirtilen şartlar çerçevesinde destek verilebilecektir: \\
\noindent  1- Yurt dışı saha çalışması, proje kapsamında ihtiyaç duyulan verilerin temin edilmesinin başka bir yolunun olmadığı durumlarda yapılabilir. Ancak, bu amaçla proje ekibine yurt dışından araştırmacı, bursiyer vb. dahil edilemez.\\
\noindent  2- Yurt dışı saha çalışmalarında araştırma sonuçlarının Türkiye'de kullanılması ve ülkemizde ihtiyaç duyulan bir husus olması zorunludur.\\
\noindent 3- Yurt dışı saha çalışmasının gerçekleştirileceği ülkelerde, başvuru öncesinde, saha çalışmasının yapılacağı yerlerin yetkili kurum/kuruluşlarından gerekli yasal/özel izinlerin alınması zorunludur. Araştırmanın yapılacağı ülkenin mevzuatı gereğince o ülkeden de Etik Kurul Onay Belgesi alınması gerekiyorsa başvuru öncesinde söz konusu belgenin de mutlaka temin edilmesi gerekmektedir.\\
\noindent 4- Yurt dışı saha çalışması için talep edilebilecek bütçe, ilgili programın destek üst limitinin \% 30'unu geçemez.\\
\noindent \textbf{Proje kapsamında yurt dışı saha çalışması yapılmasının gerekçesini ve projeye sağlayacağı katkıyı yukarıda belirtilen kurallar çerçevesinde açıklayınız:}

\begin{center}
\fbox{ %
\parbox{\textwidth}
{
% aciklama baslangic
\lipsum[9-10]
% aciklama bitis
} %
}
\end{center}


\begin{center}
    \textbf{Yurt Dışı Saha Çalışması Planı}\\
 (Sadece Sosyal ve Beşeri Bilimler alanındaki projeler için geçerlidir. Satır sayısı gerektiği kadar arttırılabilir)
\end{center}

\begin{center}
\footnotesize
\begin{tabular}{|c|c|c|c|c|c|c|c|c|c|}
\hline
\multirow{2}{*}{\rotatebox{90}{\textbf{Seyahat}}} & \multirow{2}{*}{\textbf{\begin{tabular}[c]{@{}c@{}}Nereden\\ Nereye\\ Gidileceği\end{tabular}}} & \multirow{2}{*}{\textbf{\begin{tabular}[c]{@{}c@{}}Saha\\ Çalışmasının\\ Mahiyeti (**)\end{tabular}}} & \multirow{2}{*}{\textbf{\begin{tabular}[c]{@{}c@{}}Kişi x \\ Gün\\ (***)\end{tabular}}} & \multicolumn{3}{c|}{\textbf{Uluslar Arası Ulaşım (****)}} & \multicolumn{3}{c|}{\textbf{Yurt Dışı Şehir İçi Ulaşım (****)}} \\ \cline{5-10} 
 &  &  &  & \multicolumn{1}{l|}{\textbf{\begin{tabular}[c]{@{}l@{}}Uçak/\\ Otobüs/\\ Tren/Gemi\end{tabular}}} & \textbf{\begin{tabular}[c]{@{}c@{}}Taşıt\\ Kiralama\\ (gün)\end{tabular}} & \textbf{\begin{tabular}[c]{@{}c@{}}Özel/\\ Resmi/\\ Kiralık\\ Taşıt (km)\end{tabular}} & \textbf{\begin{tabular}[c]{@{}c@{}}Toplu\\ Taşıma\\ (biniş\\ sayısı)\end{tabular}} & \textbf{\begin{tabular}[c]{@{}c@{}}Taşıt\\ Kiralama\\ (gün)\end{tabular}} & \textbf{\begin{tabular}[c]{@{}c@{}}Özel/Resmi\\ /Kiralık\\ Taşıt (km)\end{tabular}} \\ \hline
1 &  &  &  &  &  &  &  &  &  \\ \hline
2 &  &  &  &  &  &  &  &  &  \\ \hline
\multicolumn{3}{|l|}{\textbf{TOPLAM}} &  &  &  &  &  &  &  \\ \hline
\end{tabular}\\%
\end{center}
{\footnotesize
\noindent (*) Saha çalışması için farklı ülkeler ve bu ülkelerdeki farklı bölgelere yapılacak seyahatler söz konusu ise her bir seyahat için ayrı bir satır doldurulmalı ve ayrı numara verilmelidir. Aynı bölgeye farklı zamanlarda gidilecek olması durumunda da her seyahate ait bilgiler birbirini takip eden satırlara ayrı ayrı girilmelidir.\\ 
\noindent (**) Bu bölümde nerede (mahalle, okul, resmi/özel işyeri, hastane, milli/tabiat parkı, sulak alan, ormanlık alan, koruma bölgesi, doğal/tarihi sit alanı, arkeolojik kazı alanı, mağara, askeri bölge, özel bölge, tarım alanı, çiftlik, mezbaha vb.) ne yapılacağı ( anket, mülakat, örnek toplama, bilgi/belge temini, analiz vb.) belirtilir. Saha çalışmasının yeri ve yapılacak işin niteliğinin yasal/özel izin gerektirebileceği hatırlanarak TÜBİTAK ana sayfasında yer alan ``YASAL/ÖZEL İZİN BELGESİ BİLGİ NOTU ve ETİK KURUL ONAY BELGESİ BİLGİ NOTU'' nun tekrar incelenmesi önerilir. 
Yurt dışı saha çalışmasının gerçekleştirileceği ülkelerde, başvuru öncesinde, saha çalışmasının yapılacağı yerlerin yetkili kurum/kuruluşlarından gerekli yasal/özel izinlerin alınması zorunludur. Araştırmanın yapılacağı ülkenin mevzuatı gereğince o ülkeden de Etik Kurul Onay Belgesi alınması gerekiyorsa başvuru öncesinde söz konusu belgenin de mutlaka temin edilmesi gerekmektedir.\\
\noindent (***) Bu bölümde ilgili saha çalışmasına proje ekibinden kaç kişinin kaç gün süre ile katılacağı belirtilir. \\
\noindent (****) Bu bölümde Uluslar arası ve yurt dışında ilgili ülkedeki şehir içi ulaşımın hangi yolla gerçekleşeceği ilgili kısımda gidiş-dönüş olarak belirtilmelidir. Uçak/Otobüs/Tren/Gemi kısmına ilgili ulaşım aracının niteliği, şehir içiyse toplam biniş sayısı; taşıt kiralama kısmına eğer seyahat taşıt kiralama yoluyla gerçekleştirilecekse kiralanacak taşıtın niteliği ve kaç gün kiralanacağı; şayet seyahat proje ekibine ait bir özel araç ya da kuruma ait resmi araç veya kiralanacak taşıtla gerçekleşecekse yakıt giderinin hesaplanabilmesi için Özel/Resmi/Kiralık Taşıt kısmına toplam kaç kilometre mesafe kat edileceği belirtilir. 
}

\begin{center}
    \textbf{Yurt Dışı Saha Çalışması Seyahat Giderleri (03.3)}\\
(Yurt Dışı Saha Çalışması Planındaki verilerle uyumlu olacak şekilde doldurulur ve bütçelendirilir)
\end{center}


\begin{center}
\footnotesize
    \begin{tabular}{|l|c|c|c|c|c|c|}
\hline
 & \textbf{\begin{tabular}[c]{@{}c@{}}Kişi\\ Sayısı\end{tabular}} & \textbf{\begin{tabular}[c]{@{}c@{}}Seyahat\\ Adedi (kez)\end{tabular}} & \textbf{\begin{tabular}[c]{@{}c@{}}Toplam\\ Gün\end{tabular}} & \textbf{\begin{tabular}[c]{@{}c@{}}Taşıt Kirası\\ (ücret x gün)\end{tabular}} & \textbf{\begin{tabular}[c]{@{}c@{}}Toplam\\ Katedilecek Yol\\ (km)\end{tabular}} & \textbf{TOPLAM} \\ \hline
\textbf{\begin{tabular}[c]{@{}l@{}}Uluslar Arası Seyahat \\ (uçak/otobüs/tren)\end{tabular}} &  &  & \cellcolor[HTML]{656565}{\color[HTML]{656565} } & \cellcolor[HTML]{656565}{\color[HTML]{656565} } & \cellcolor[HTML]{656565}{\color[HTML]{656565} } &  \\ \hline
\textbf{\begin{tabular}[c]{@{}l@{}}Yurt Dışında Şehir İçi Toplu Taşıma\\ (otobüs/tren/metro vb.)\end{tabular}} &  &  & \cellcolor[HTML]{656565} & \cellcolor[HTML]{656565} & \cellcolor[HTML]{656565} &  \\ \hline
\textbf{\begin{tabular}[c]{@{}l@{}}Özel/Resmi/Kiralık Taşıt ile\\ Seyahat (*)\end{tabular}} & \cellcolor[HTML]{656565} & \cellcolor[HTML]{656565} & \cellcolor[HTML]{656565} & \cellcolor[HTML]{656565} &  &  \\ \hline
\textbf{Taşıt Kirası Gideri} & \cellcolor[HTML]{656565} & \cellcolor[HTML]{656565} &  &  & \cellcolor[HTML]{656565} &  \\ \hline
\textbf{\begin{tabular}[c]{@{}l@{}}Gündelik (**)\\ (proje ekibi)\end{tabular}} &  &  &  & \cellcolor[HTML]{656565} & \cellcolor[HTML]{656565} &  \\ \hline
\textbf{\begin{tabular}[c]{@{}l@{}}Konaklama (**)\\ (proje ekibi)\end{tabular}} &  &  &  & \cellcolor[HTML]{656565} & \cellcolor[HTML]{656565} &  \\ \hline
\multicolumn{6}{|l|}{\textbf{TOPLAM (TL)}} &  \\ \hline
\end{tabular}\\%
\end{center}
{\footnotesize
\noindent (*) Özel/Resmi/KiralıkTaşıt ile yapılan seyahatlerde her 100 km. için 6 litre kurşunsuz benzin ücreti ödeneceği dikkate alınarak hesaplanır.\\
\noindent (**) Yurt dışı gündelik ve konaklama bedelleri 6245 sayılı Harcırah Kanunu hükümleri uyarınca hesaplanıp bütçelendirilir.

}

\begin{center}
\footnotesize
    \begin{tabular}{|l|l|}
\hline
\multicolumn{2}{|c|}{\textbf{\begin{tabular}[c]{@{}c@{}}\normalsize Seyahat Giderleri\\ 
Saha Çalışması Dışındaki Faaliyetler İçin Yapılacak Olan Yurt İçi / Yurt Dışı Seyahatler\\  Bilimsel Toplantılara Katılma, Çalışma Ziyaretleri vb. Faaliyetler)(*) (03.4)\end{tabular}}} \\ \hline
 & \multicolumn{1}{c|}{\textbf{TOPLAM (TL)}} \\ \hline
\textbf{Yurt İçi / Yurt Dışı Seyahat} &  \\ \hline
\end{tabular}%
\end{center}

\begin{center}
\footnotesize
    \begin{tabular}{|l|l|l|l|}
\hline
\multicolumn{4}{|c|}{\normalsize \textbf{Bursiyer Ücretleri (*) (05.4)}} \\ \hline
\multicolumn{1}{|c|}{\begin{tabular}[c]{@{}c@{}}\textbf{Niteliği}\\ (Lisans/Y. Lisans/Doktora/Doktora Sonrası Araştırmacı)\end{tabular}} & \multicolumn{1}{c|}{\textbf{\begin{tabular}[c]{@{}c@{}}Projede Yer Alma Süresi\\ (ay)\end{tabular}}} & \multicolumn{1}{c|}{\textbf{\begin{tabular}[c]{@{}c@{}}Burs Miktarı\\ (TL/ay)\end{tabular}}} & \multicolumn{1}{c|}{\textbf{\begin{tabular}[c]{@{}c@{}}Toplam\\ (TL)\end{tabular}}} \\ \hline
 &  &  &  \\ \hline
 &  &  &  \\ \hline
\multicolumn{3}{|r|}{\textbf{TOPLAM}} &  \\ \hline
\end{tabular}\\ %
\end{center}
{\footnotesize
\noindent (*) Bursiyer(ler)in projede yapacağı faaliyet ile ilgili ayrıntılı bilgi ek sayfada verilir. Projede yer alacak bursiyer(ler)in eğitim alanlarının veya tez konularının proje konusunun ilgili olduğu alan(lar)da olması beklenmektedir. Bursiyerler aynı anda birden fazla projede yer alamazlar. BİDEB'den tam burs alan Lisansüstü ve Doktora Sonrası Bursiyerler için projeden de ek burs ödemesi yapılabilir. BİDEB'den kısmi burs alanlara projeden ayrıca burs ödemesi yapılmaz.\\
\noindent Lisans bursiyeri olmak için Türkiye'de kurulu bir yükseköğretim kurumunun lisans programında 3.sınıf ve üzeri öğrencisi olmak, herhangi bir kuruluşta ücret karşılığı çalışmamak ve üniversitenin not sistemi esas olmak üzere hazırlık hariç önceki yılların ağırlıklı genel not ortalamasının en az 4 üzerinden 2,5 veya 100 üzerinden 65 olması veya bölümünde hazırlık sınıfı hariç önceki yıllara ait ağırlıklı genel not ortalamasında ilk \%20 lik dilime girilmiş olması gerekmektedir. Bir projede aynı anda en fazla 4 lisans öğrencisi bursiyer olarak yer alabilir.
}

\begin{center}
    \textbf{Burs Miktarı Üst Sınırları}\\
\begin{tabular}{|l|c|c|}
\hline
\multicolumn{1}{|c|}{\textbf{Niteliği}} & \textbf{Ücret Karşılığı Çalışmıyor İse} & \textbf{Ücretli Çalışıyor İse} \\ \hline
Lisans Öğrencisi & 750.-TL/ay & ------------ \\ \hline
Yüksek Lisans Öğrencisi & \begin{tabular}[c]{@{}c@{}}3.000.-TL/ay\\ (BİDEB bursiyeri olması halinde 500 TL ilave edilir)\end{tabular} & 550.-TL/ay \\ \hline
Doktora Öğrencisi & \begin{tabular}[c]{@{}c@{}}3.500.-TL/ay\\ (BİDEB bursiyeri olması halinde 1.000 TL ilave edilir)\end{tabular} & 650.-TL/ay \\ \hline
Doktora Sonrası Araştırmacı & \begin{tabular}[c]{@{}c@{}}4.500.-TL/ay\\ (BİDEB bursiyeri olması halinde 1.500 TL ilave edilir)\end{tabular} & ------------ \\ \hline
\end{tabular}%
\end{center}


\begin{center}
\footnotesize
\begin{tabular}{|l|l|l|l|l|}
\hline
\multicolumn{5}{|c|}{\normalsize\textbf{Geçici İşçi Ücretleri (Yardımcı Personel) (*) (01.3)}} \\ \hline
\textbf{Adı Soyadı} & \multicolumn{1}{c|}{\textbf{Nitelik}} & \multicolumn{1}{c|}{\textbf{Görev Süresi (ay)}} & \multicolumn{1}{c|}{\textbf{\begin{tabular}[c]{@{}c@{}}Aylık Ücret\\ (TL/ay)\end{tabular}}} & \multicolumn{1}{c|}{\textbf{\begin{tabular}[c]{@{}c@{}}Toplam\\ (TL)\end{tabular}}} \\ \hline
 &  &  &  &  \\ \hline
 &  &  &  &  \\ \hline
 & \multicolumn{1}{r|}{\textbf{}} &  &  &  \\ \hline
\end{tabular}\\ %
\end{center}
{\footnotesize
\noindent (*) Projede görev yapacak teknisyen, laborant, mühendis, vb. yardımcı personelin projede yapacağı çalışma ile ilgili ayrıntılı bilgi ek sayfada verilir. \\
\noindent Herhangi bir yerde tam zamanlı olarak çalışan kimseler projelerde yardımcı personel olarak yer alamaz ve bu kişilere projeden ödeme yapılamaz.\\
\noindent Yardımcı personelin herhangi bir yerde çalışmıyor olması durumunda; kendisine ödenecek düzenli ücret projeye brüt olarak yansıtılır. Bu durumda çalışan kişi ile ilgili vergi, sosyal güvenlik kesintileri ve mevzuatların öngördüğü yasal zorunlulukların yerine getirilmesi yürütücü sorumluluğundadır. \\
\noindent Herhangi bir yerde çalışmayan yardımcı personele yapılacak ödemelerin miktarı; projenin yürütüldüğü kurumda çalışan ve yapılacak işin muadili ya da yakın görevdeki kişinin maaşının rayiç bedeli alınarak yürütücü tarafından belirlenir.
}